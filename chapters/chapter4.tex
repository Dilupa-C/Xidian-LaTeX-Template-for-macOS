\chapter{评测与分析}\label{chap:evaluation}
该章节演示如何记录实验设置、指标与结论。所有数字均为示意,目的是让用户了解排版结构与交叉引用写法。\parencite{evaluation-handbook}

\section{实验设置}
假设示例系统在三组不同策略下运行,记录样本量、时间跨度与硬件环境即可。为了突出关键参数,可以插入简化公式:
\begin{equation}
  \mathrm{Score} = \alpha \cdot \mathrm{Engagement} + (1-\alpha) \cdot \mathrm{Accuracy}, \quad \alpha = 0.6.
\end{equation}

\section{结果展示}
文本中描述趋势,再配合表格或图形。图~\ref{fig:evaluation-placeholder} 依旧使用 `figures/ch4/xdulogo.pdf` 作为占位图,提醒读者在公开仓库中不要存放敏感数据。
\begin{figure}[htbp]\centering
  \includegraphics[width=0.4\textwidth,height=0.4\textwidth,keepaspectratio]{figures/ch4/xdulogo.pdf}
  \caption{评测章节的占位图,可替换为真实结果\label{fig:evaluation-placeholder}}
\end{figure}

\section{讨论}
讨论部分可以回溯需求与实现章节的假设,分析失败案例或潜在风险,再说明可复现步骤。使用条列或小节能够增强可读性。
