\chapter{引言}\label{chap:intro}
本章用于演示在 XDUTS 模板中撰写绪论时的结构安排,包括研究背景、目标与文章组织方式。为了降低泄密风险,全文采用虚构的“自适应学习平台”作为案例,通过描述常见痛点与解决思路来展示排版流程。\parencite{template-guide}

\section{研究背景与动机}
随着混合式教学与远程协作的普及,学生与教师需要一套跨平台一致的排版模版,以避免因字体或宏包版本差异导致的版式偏差。本模板通过明确 MacTeX 版本、字体依赖和 latexmk 脚本设置,保证编译结果一致。\parencite{adaptive-survey}

\section{示例研究目标}
示例课题的目标是构建一个可复用的排版工作流,用于记录需求分析、实现方案、实验评估与附录脚本。各章节采用占位文字描述对应模块,提醒读者在真实写作时保持结构清晰、引用准确。

\section{文章结构安排}
第\ref{chap:modeling}章到第\ref{chap:discussion}章依次介绍需求建模、系统实现、评测流程与风险提示,结论章总结经验并提出后续工作设想。附录部分预留脚本与额外图表示例,便于扩展。
