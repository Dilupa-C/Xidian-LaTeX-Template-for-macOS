\chapter{风险与展望}\label{chap:discussion}
最后一章用于总结风险清单并给出未来工作设想。示例内容完全虚构,旨在提醒作者在最终提交前完成自查与计划梳理。

\section{风险清单}
\begin{itemize}
  \item 安全:确保示例数据与脚本不包含个人信息;
  \item 可维护性:记录依赖版本并在仓库附上操作说明;
  \item 复现性:提供脚本或伪代码,使同伴可以独立验证。
\end{itemize}

\section{未来工作}
可以提出若干扩展方向,如引入更细颗粒的自适应策略、加强可解释性分析或将模板与 CI/CD 流水线结合等。

\begin{figure}[htbp]\centering
  \includegraphics[width=0.4\textwidth,height=0.4\textwidth,keepaspectratio]{figures/ch7/xdulogo.pdf}
  \caption{未来工作章节的占位插图\label{fig:future-work-placeholder}}
\end{figure}
