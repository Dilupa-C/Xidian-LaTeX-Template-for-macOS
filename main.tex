\documentclass{xdupgthesis}
\usepackage{tabularray}%最好用的表格排版宏包
\usepackage{subcaption}%排版子图
\usepackage{xcolor}
%\UseTblrLibrary{booktabs}
\usepackage{amssymb}
\usepackage{amsthm}
\usepackage{amsmath}
\usepackage{mathrsfs}
\usepackage{arydshln}  %矩阵中的虚线
\usepackage{bbding}
\usepackage{tikz}
\usepackage{enumitem}
\usepackage{bm}
%\usepackage{natbib}
%\usepackage{gbt7714}
%\graphicspath{figures/}
\xdusetup{
style = {
	cjk-font = win, 					%中文字体
%	latin-font = tac,					%英文字体
	title-bold-math = true,				%标题数学字体加粗
	language = zh,  					%论文语言
	customize-edubg = false,
	customize-resresult = false,
	customize-los = false,
	title-row-los = false,				%仅第一页显示符号对照表标题
	colspec-los = {Q[l,t]X[l,t]}, 		%设置符号对照表格式
	customize-loa = false,
	title-row-loa = false,				%仅第一页显示缩略语对照表标题
	colspec-loa = {X[l,t]X[3,t]X[2,t]},	%设置缩略语对照表格式
	bib-backend = bibtex,
	customize-edubg = false,
	customize-resresult = false,
	font-type = file,
	font-path = {Font},
%	remove-page = {封面,题名页,声明页,目录,参考文献,致谢,作者简介}, %查重格式,由于正文引用了附录,因此单独删除附录
%	remove-page = {致谢}, %明审格式
%	remove-header = true, %移除页眉
%	remove-footer = true, %移除页脚
	},
info = {
graduate-type = {博士},
degree-type = {学术},
degree = {工学博士},
title = {自适应学习平台排版流程示例},
title* = {Sample Workflow for Adaptive Learning Theses},
department = {信息工程学院},
major = {智能系统与工程},
major* = {Intelligent Systems Engineering},
sub-major = {数字学习与系统设计},
submit-date = {2024-9},
author = {西电示例同学},
author* = {Sample Student},
supervisor = {示例导师},
supervisor* = {Sample Advisor},
supv-title = {副教授},
supv-title* = {Associate Professor},
student-id = {2024000000},
clc = {TP271},
abstract = {chapters/abstract-zh.tex},
abstract* = {chapters/abstract-en.tex},
keywords = {模板演示,自适应学习,排版流程,编译环境,字体依赖},
keywords* = {template demonstration, adaptive learning, workflow, build environment, font setup},
los = {chapters/los.tex},	%符号对照表
loa = {chapters/loa.tex},	%缩略语对照表
bib-resource = {references.bib},
appendix = {{chapters/appendixA.tex}},%,{chapters/appendixB.tex}
acknowledgements = {chapters/acknowledgements.tex},
bio = {chapters/bio.tex},
}
}
%--------------定义新的定理环境格式---------------
\newtheoremstyle{mytheorem}{}{}{}{\parindent}{\bfseries}{.}{.5em}{}
\theoremstyle{mytheorem}
\newtheorem{theorem}{定理}[chapter]
\newtheorem{corollary}{推论}[chapter]
\newtheorem{lemma}{引理}[chapter]
\newtheorem{definition}{定义}[chapter]
\newtheorem{proposition}{命题}[chapter]
\newtheorem{property}{性质}[chapter]
\newtheorem{assumption}{假设}[chapter]
\newtheorem{example}{例}[section]
%--------------定义新的注记环境格式---------------
\newtheoremstyle{myremark}{}{}{}{\parindent}{\bfseries}{.}{.5em}{}
\theoremstyle{myremark}
\newtheorem*{remark}{注}
%--------------定义新的证明环境格式---------------
\newtheoremstyle{myproof}{}{}{}{\parindent}{\bfseries}{:}{.5em}{}
\theoremstyle{myproof}
\newtheorem*{Proof}{证明}
%--------------定义新的表格横线格式---------------
\NewTableCommand\toprule{\hline[0.08em]}
\NewTableCommand\midrule{\hline[0.05em]}
\NewTableCommand\bottomrule{\hline[0.08em]}
%--------------定义带圈数字-----------------------
\newcommand*{\circled}[1]{\lower.7ex\hbox{\tikz\draw (0pt, 0pt)%
		circle (.5em) node {\makebox[1em][c]{\small #1}};}} %圆圈数字
\begin{document}
\chapter{引言}\label{chap:intro}
本章用于演示在 XDUTS 模板中撰写绪论时的结构安排,包括研究背景、目标与文章组织方式。为了降低泄密风险,全文采用虚构的“自适应学习平台”作为案例,通过描述常见痛点与解决思路来展示排版流程。\parencite{template-guide}

\section{研究背景与动机}
随着混合式教学与远程协作的普及,学生与教师需要一套跨平台一致的排版模版,以避免因字体或宏包版本差异导致的版式偏差。本模板通过明确 MacTeX 版本、字体依赖和 latexmk 脚本设置,保证编译结果一致。\parencite{adaptive-survey}

\section{示例研究目标}
示例课题的目标是构建一个可复用的排版工作流,用于记录需求分析、实现方案、实验评估与附录脚本。各章节采用占位文字描述对应模块,提醒读者在真实写作时保持结构清晰、引用准确。

\section{文章结构安排}
第\ref{chap:modeling}章到第\ref{chap:discussion}章依次介绍需求建模、系统实现、评测流程与风险提示,结论章总结经验并提出后续工作设想。附录部分预留脚本与额外图表示例,便于扩展。

\chapter{需求建模示例}\label{chap:modeling}
本章演示如何用模板描述虚构项目的需求分析。内容包括场景化描述、信息流示意与符号定义,所有数据均为占位信息,仅用于展示排版效果。\parencite{workflow-tools}

\section{场景化描述}
示例项目将角色分为教师、学生与教学支持团队,通过访谈收集痛点并整理成需求表。作者可在此处引入定义、假设或表格,以说明关键变量及其约束。

\section{信息流示意}
图\ref{fig:logo-demo} 使用 `figures/ch2/xdulogo.pdf` 作为占位图,展示章节化管理插图的方式。将真实图片放入对应目录后,只需替换文件名即可。

\begin{figure}[htbp]\centering
  \includegraphics[width=0.4\textwidth,height=0.4\textwidth,keepaspectratio]{figures/ch2/xdulogo.pdf}
  \caption{需求阶段的信息流示意\label{fig:logo-demo}}
\end{figure}

\section{符号与记号}
示例系统在迭代 $k$ 时刻的状态为 $\bm{s}_k$,策略参数记为 $\bm{\theta}$。定义上层调度函数 $f(\cdot)$ 与噪声项 $\bm{\eta}_k$,即可写出占位方程:
\begin{equation}
  \bm{s}_{k+1} = f(\bm{s}_k, \bm{\theta}) + \bm{\eta}_k,
\end{equation}
并在符号表中登记符号含义,方便后续引用。\parencite{style-guide}

\chapter{系统实现示例}\label{chap:implementation}
本章给出占位的技术实现流程,包括模块划分、配置示例与脚本调度方式。请在真实项目中将相关描述替换为实际的工程内容。\parencite{workflow-tools}

\section{模块划分}
参考“数据采集—建模推理—可视化”三层结构列出核心模块。可通过枚举或流程图说明模块之间的接口,帮助读者理解系统边界。

\section{配置样例}
表~\ref{tab:config} 展示了一个最小化配置项,说明如何使用 `tabularray` 绘制带标题的表格。
\begin{table}[htbp]
  \centering
  \begin{tblr}{colspec={lX[2]X[2]}}
    \toprule
    组件 & 示例参数 & 说明 \\
    \midrule
    数据接口 & `collector.yaml` & 定义数据源、采样频率与脱敏策略。 \\
    推理引擎 & `engine.toml` & 控制训练轮数、随机种子与日志级别。 \\
    可视化 & `report.json` & 记录图表主题、配色以及导出格式。 \\
    \bottomrule
  \end{tblr}
  \caption{示例配置表}
  \label{tab:config}
\end{table}

\section{脚本示例}
可以通过 `lstlisting` 或 `verbatim` 展示运行命令:
\begin{verbatim}
python tools/run_pipeline.py --config configs/demo.yaml
\end{verbatim}
当段落需要引用其他章节结果时,可使用“如第~\ref{chap:evaluation} 章所述”保持连贯。

\section{界面占位图}
图~\ref{fig:impl-demo} 演示如何引用 `figures/ch3/xdulogo.pdf`。实际使用时,将图片替换为真实架构或界面草图即可。
\begin{figure}[htbp]\centering
  \includegraphics[width=0.4\textwidth,height=0.4\textwidth,keepaspectratio]{figures/ch3/xdulogo.pdf}
  \caption{系统实现章节的示意插图\label{fig:impl-demo}}
\end{figure}

\chapter{评测与分析}\label{chap:evaluation}
该章节演示如何记录实验设置、指标与结论。所有数字均为示意,目的是让用户了解排版结构与交叉引用写法。\parencite{evaluation-handbook}

\section{实验设置}
假设示例系统在三组不同策略下运行,记录样本量、时间跨度与硬件环境即可。为了突出关键参数,可以插入简化公式:
\begin{equation}
  \mathrm{Score} = \alpha \cdot \mathrm{Engagement} + (1-\alpha) \cdot \mathrm{Accuracy}, \quad \alpha = 0.6.
\end{equation}

\section{结果展示}
文本中描述趋势,再配合表格或图形。图~\ref{fig:evaluation-placeholder} 依旧使用 `figures/ch4/xdulogo.pdf` 作为占位图,提醒读者在公开仓库中不要存放敏感数据。
\begin{figure}[htbp]\centering
  \includegraphics[width=0.4\textwidth,height=0.4\textwidth,keepaspectratio]{figures/ch4/xdulogo.pdf}
  \caption{评测章节的占位图,可替换为真实结果\label{fig:evaluation-placeholder}}
\end{figure}

\section{讨论}
讨论部分可以回溯需求与实现章节的假设,分析失败案例或潜在风险,再说明可复现步骤。使用条列或小节能够增强可读性。

\chapter{扩展案例}\label{chap:extensions}
该章节展示如何添加可选的扩展内容,例如多终端协作、对外接口或部署脚本。所有描述均为模板化占位,请根据实际情况替换。\parencite{template-guide}

\section{接口清单}
可以列出需要集成的系统、通信协议与安全约束。若接口较多,可拆分为子小节或附录。

\section{用户研究}
若项目包含用户测试,可在此概述抽样策略与伦理注意事项,但不写入真实个人信息。可指向附录获取更详细的问卷模板。

\section{示例占位图}
\begin{figure}[htbp]\centering
  \includegraphics[width=0.4\textwidth,height=0.4\textwidth,keepaspectratio]{figures/ch5/xdulogo.pdf}
  \caption{扩展章节的占位插图\label{fig:extension-placeholder}}
\end{figure}

\chapter{协作与复现建议}\label{chap:collab}
本章总结多人协作、数据脱敏与复现步骤。保持这些占位内容可以帮助团队快速理解模板结构,然后根据项目需求加以扩展。\parencite{workflow-tools}

\section{环境准备}
强调统一字体、MacTeX 版本以及 `latexmkrc` 配置的重要性。可列出常用命令,提醒成员在提交前执行 `latexmk -C` 清理中间文件。

\section{数据管理}
公开仓库仅保留合成或匿名化数据,真实数据应存放在受控存储中。若需要共享脚本,可在附录中提供伪代码或命令示例。

\begin{figure}[htbp]\centering
  \includegraphics[width=0.4\textwidth,height=0.4\textwidth,keepaspectratio]{figures/ch6/xdulogo.pdf}
  \caption{数据管理流程示意\label{fig:dataflow-placeholder}}
\end{figure}

\chapter{风险与展望}\label{chap:discussion}
最后一章用于总结风险清单并给出未来工作设想。示例内容完全虚构,旨在提醒作者在最终提交前完成自查与计划梳理。

\section{风险清单}
\begin{itemize}
  \item 安全:确保示例数据与脚本不包含个人信息;
  \item 可维护性:记录依赖版本并在仓库附上操作说明;
  \item 复现性:提供脚本或伪代码,使同伴可以独立验证。
\end{itemize}

\section{未来工作}
可以提出若干扩展方向,如引入更细颗粒的自适应策略、加强可解释性分析或将模板与 CI/CD 流水线结合等。

\begin{figure}[htbp]\centering
  \includegraphics[width=0.4\textwidth,height=0.4\textwidth,keepaspectratio]{figures/ch7/xdulogo.pdf}
  \caption{未来工作章节的占位插图\label{fig:future-work-placeholder}}
\end{figure}

\chapter{结论与展望}\label{chap:conclusion}
示例论文演示了如何在 XDUTS 模板中组织章节、插入图表和管理参考文献。主要经验包括:统一环境、封装 latexmk 脚本以及对敏感信息进行脱敏处理。未来可以在此模板基础上加入自动化检查或扩展到其他操作系统。

\end{document}
